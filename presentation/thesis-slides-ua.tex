%&latex
% Copyright 2011 Zoresvit (c) <zoresvit@gmail.com>

\documentclass[10pt, ucs]{beamer}
\usefonttheme[onlymath]{serif}
\usetheme{zslides}
\usepackage{xcolor}
\usepackage{tikz}

\graphicspath{{images/}}
\setbeamercolor{alerted text}{fg=red!50!black}

%\usepackage{pgfpages}
% \pgfpagesuselayout{2 on 1}[a4paper,border shrink=5mm]
% \mode<handout>{\setbeamercolor{background canvas}{bg=black!5}}

\addtobeamertemplate{title page}{%
\vspace*{2ex}\large\centering Харківський національний університет радіоелектроніки\\
}{}

\newcommand{\worktitle}{Аналіз криптографічних властивостей симетричних шифрів \\[2ex]}
\newcommand{\workauthors}{Руслан Кіянчук}


\title[Криптографічні властивості симетричних шифрів]{\worktitle}
%\author[\workauthors]{
%\texorpdfstring{Руслан Кіянчук\\[2ex] \scriptsize\url{ruslan.kiyanchuk@gmail.com}}{\workauthors} \\[4ex]
%\begin{flushleft}
%    \hspace*{10em}
%    \normalsize Науковий керівник: \hspace*{10ex} Халімов Г. З. \\[0.8ex]
%    \hspace*{6.9em}
%    \normalsize Консультант: \hspace*{15.8ex} Олійников Р. В.
%\end{flushleft}
%}

\subtitle{Бакалаврська робота}
\institute[ХНУРЕ]{}
\date[Харків 2013]{\normalsize Харків 2013}

\hypersetup{
pdftitle = {\worktitle},
pdfauthor = {\workauthors}, 
pdfkeywords = {}
}


\begin{document}
\maketitle

\section{Актуальність}
\begin{frame}{Актуальність}{Нагальні проблеми}
    \small
    \begin{block}{}
        \begin{itemize}
            \item Сучасні криптоалгоритми базуються на складній математиці.
            \item Реальні системи безпеки часто є вразливими із-за некоректного
                використання та помилок реалізації.
        \end{itemize}
    \end{block}
    \alert{стандарт GSM:}
    \begin{description}
        \item[A5/1] можливо зламати за секунди
            (використовуючи~rainbow~таблиці).
    \end{description}
    \alert{Супутникові телефони:}
    \begin{description}
        \item[GMR-1] базується на A5/2 (виведений з експлуатації з 2007~р.);
        \item[GMR-2] неопублікований алгоритм (атака потребує $65$~б гами).
    \end{description}
    \alert{Безпровідний інтернет:}
    \begin{description}
        \item [WEP] необхідно $80000$ пакетів для проведення криптоаналізу;
        \item[E0] атака вимагає $2^{38}$ операцій та $2^{23.8}$~фрагментів
            даних.
    \end{description}
    \begin{block}{}
        Необхідно проводити детальний аналіз та відкриту експертизу шифрів
        перед застосуванням у реальних системах безпеки.
    \end{block}
\end{frame}

\begin{frame}[shrink=1.2]{Актуальність}{Напрями розвитку криптографії}
    \footnotesize
    \begin{block}{Актуальність ГОСТ~28147}
        \begin{itemize}
            \item прийнятий у 1989~р. стандарт шифрування ГОСТ широко
                застосовується в Україні та країнах СНД;
            \item запропонований для стандартизації в ISO у 2010 р;
            \item перспективний для використання у мало-ресурсній криптографії;
            \item відсутня оцінка стійкості до алгебраїчного криптоаналізу.
        \end{itemize}
    \end{block}
\end{frame}


\section{Алгебраїчний криптоаналіз  ГОСТ~28147}
\begin{frame}{Алгебраїчний криптоаналіз}
    \begin{block}{Клод Шеннон}
        ``Злам стійкого шифру має потребувати такий же обсяг обчислень, що і
        вирішення системи рівнянь від багатьох невідомих''.
    \end{block}
    \begin{minipage}[t]{0.5\textwidth}
        \begin{block}{Основні принципи}
            \begin{enumerate}
                \item криптоалгоритм описується системою рівнянь другого
                    степеня від багатьох змінних;
                \item за наявності відкритих повідомлень та шифротекстів,
                    система рівнянь вирішується для знаходження бітів ключа.
            \end{enumerate}
        \end{block}
    \end{minipage}%
    \hspace*{1ex}
    \begin{minipage}[t]{0.5\textwidth}
        \begin{figure}[htbp]
            \centering
            % Graphic for TeX using PGF
% Title: /home/zoresvit/Documents/diploma/bachelor-thesis/images/gost.dia
% Creator: Dia v0.97.1
% CreationDate: Wed May  9 22:47:17 2012
% For: zoresvit
% \usepackage{tikz}
% The following commands are not supported in PSTricks at present
% We define them conditionally, so when they are implemented,
% this pgf file will use them.
\ifx\du\undefined
  \newlength{\du}
\fi
\setlength{\du}{15\unitlength}
\begin{tikzpicture}[every node/.style={scale=0.8}]
\pgftransformxscale{1.000000}
\pgftransformyscale{-1.000000}
\definecolor{dialinecolor}{rgb}{0.000000, 0.000000, 0.000000}
\pgfsetstrokecolor{dialinecolor}
\definecolor{dialinecolor}{rgb}{1.000000, 1.000000, 1.000000}
\pgfsetfillcolor{dialinecolor}
\definecolor{dialinecolor}{rgb}{1.000000, 1.000000, 1.000000}
\pgfsetfillcolor{dialinecolor}
\fill (16.000000\du,2.000000\du)--(16.000000\du,4.000000\du)--(28.000000\du,4.000000\du)--(28.000000\du,2.000000\du)--cycle;
\pgfsetlinewidth{0.100000\du}
\pgfsetdash{}{0pt}
\pgfsetdash{}{0pt}
\pgfsetmiterjoin
\definecolor{dialinecolor}{rgb}{0.000000, 0.000000, 0.000000}
\pgfsetstrokecolor{dialinecolor}
\draw (16.000000\du,2.000000\du)--(16.000000\du,4.000000\du)--(28.000000\du,4.000000\du)--(28.000000\du,2.000000\du)--cycle;
% setfont left to latex
\definecolor{dialinecolor}{rgb}{0.000000, 0.000000, 0.000000}
\pgfsetstrokecolor{dialinecolor}
\node at (22.000000\du,3.000000\du){INPUT};
\definecolor{dialinecolor}{rgb}{1.000000, 1.000000, 1.000000}
\pgfsetfillcolor{dialinecolor}
\fill (16.000000\du,12.000000\du)--(16.000000\du,14.000000\du)--(28.000000\du,14.000000\du)--(28.000000\du,12.000000\du)--cycle;
\pgfsetlinewidth{0.100000\du}
\pgfsetdash{}{0pt}
\pgfsetdash{}{0pt}
\pgfsetmiterjoin
\definecolor{dialinecolor}{rgb}{0.000000, 0.000000, 0.000000}
\pgfsetstrokecolor{dialinecolor}
\draw (16.000000\du,12.000000\du)--(16.000000\du,14.000000\du)--(28.000000\du,14.000000\du)--(28.000000\du,12.000000\du)--cycle;
% setfont left to latex
\definecolor{dialinecolor}{rgb}{0.000000, 0.000000, 0.000000}
\pgfsetstrokecolor{dialinecolor}
\node at (22.000000\du,13.000000\du){OUTPUT};
\definecolor{dialinecolor}{rgb}{1.000000, 1.000000, 1.000000}
\pgfsetfillcolor{dialinecolor}
\fill (30.000000\du,2.000000\du)--(30.000000\du,4.000000\du)--(36.000000\du,4.000000\du)--(36.000000\du,2.000000\du)--cycle;
\pgfsetlinewidth{0.100000\du}
\pgfsetdash{}{0pt}
\pgfsetdash{}{0pt}
\pgfsetmiterjoin
\definecolor{dialinecolor}{rgb}{0.000000, 0.000000, 0.000000}
\pgfsetstrokecolor{dialinecolor}
\draw (30.000000\du,2.000000\du)--(30.000000\du,4.000000\du)--(36.000000\du,4.000000\du)--(36.000000\du,2.000000\du)--cycle;
% setfont left to latex
\definecolor{dialinecolor}{rgb}{0.000000, 0.000000, 0.000000}
\pgfsetstrokecolor{dialinecolor}
\node at (33.000000\du,3.000000\du){SUBKEY};
\pgfsetlinewidth{0.100000\du}
\pgfsetdash{}{0pt}
\pgfsetdash{}{0pt}
\pgfsetbuttcap
{
\definecolor{dialinecolor}{rgb}{0.000000, 0.000000, 0.000000}
\pgfsetfillcolor{dialinecolor}
% was here!!!
\definecolor{dialinecolor}{rgb}{0.000000, 0.000000, 0.000000}
\pgfsetstrokecolor{dialinecolor}
\draw (26.500000\du,4.500000\du)--(26.500000\du,8.000000\du);
}
\pgfsetlinewidth{0.100000\du}
\pgfsetdash{}{0pt}
\pgfsetdash{}{0pt}
\pgfsetmiterjoin
\pgfsetbuttcap
{
\definecolor{dialinecolor}{rgb}{0.000000, 0.000000, 0.000000}
\pgfsetfillcolor{dialinecolor}
% was here!!!
{\pgfsetcornersarced{\pgfpoint{0.000000\du}{0.000000\du}}\definecolor{dialinecolor}{rgb}{0.000000, 0.000000, 0.000000}
\pgfsetstrokecolor{dialinecolor}
\draw (33.000000\du,4.000000\du)--(33.000000\du,5.500000\du)--(26.754900\du,5.500000\du)--(26.754900\du,5.500000\du);
}}
\pgfsetlinewidth{0.100000\du}
\pgfsetdash{}{0pt}
\pgfsetdash{}{0pt}
\pgfsetmiterjoin
\definecolor{dialinecolor}{rgb}{1.000000, 1.000000, 1.000000}
\pgfsetfillcolor{dialinecolor}
\fill (24.800000\du,6.500000\du)--(24.800000\du,7.500000\du)--(25.800000\du,7.500000\du)--(25.800000\du,6.500000\du)--cycle;
\definecolor{dialinecolor}{rgb}{0.000000, 0.000000, 0.000000}
\pgfsetstrokecolor{dialinecolor}
\draw (24.800000\du,6.500000\du)--(24.800000\du,7.500000\du)--(25.800000\du,7.500000\du)--(25.800000\du,6.500000\du)--cycle;
\pgfsetlinewidth{0.100000\du}
\pgfsetdash{}{0pt}
\pgfsetdash{}{0pt}
\pgfsetbuttcap
{
\definecolor{dialinecolor}{rgb}{0.000000, 0.000000, 0.000000}
\pgfsetfillcolor{dialinecolor}
% was here!!!
\definecolor{dialinecolor}{rgb}{0.000000, 0.000000, 0.000000}
\pgfsetstrokecolor{dialinecolor}
\draw (25.300000\du,6.500000\du)--(25.300000\du,7.500000\du);
}
\pgfsetlinewidth{0.100000\du}
\pgfsetdash{}{0pt}
\pgfsetdash{}{0pt}
\pgfsetbuttcap
{
\definecolor{dialinecolor}{rgb}{0.000000, 0.000000, 0.000000}
\pgfsetfillcolor{dialinecolor}
% was here!!!
\definecolor{dialinecolor}{rgb}{0.000000, 0.000000, 0.000000}
\pgfsetstrokecolor{dialinecolor}
\draw (24.800000\du,7.000000\du)--(25.800000\du,7.000000\du);
}
% setfont left to latex
\definecolor{dialinecolor}{rgb}{0.000000, 0.000000, 0.000000}
\pgfsetstrokecolor{dialinecolor}
\node[anchor=west] at (27.200000\du,1.500000\du){0};
% setfont left to latex
\definecolor{dialinecolor}{rgb}{0.000000, 0.000000, 0.000000}
\pgfsetstrokecolor{dialinecolor}
\node[anchor=west] at (20.500000\du,1.500000\du){32 31};
% setfont left to latex
\definecolor{dialinecolor}{rgb}{0.000000, 0.000000, 0.000000}
\pgfsetstrokecolor{dialinecolor}
\node[anchor=west] at (15.500000\du,1.500000\du){63};
% setfont left to latex
\definecolor{dialinecolor}{rgb}{0.000000, 0.000000, 0.000000}
\pgfsetstrokecolor{dialinecolor}
\node[anchor=west] at (35.200000\du,1.500000\du){0};
% setfont left to latex
\definecolor{dialinecolor}{rgb}{0.000000, 0.000000, 0.000000}
\pgfsetstrokecolor{dialinecolor}
\node[anchor=west] at (29.500000\du,1.500000\du){31};
\pgfsetlinewidth{0.100000\du}
\pgfsetdash{}{0pt}
\pgfsetdash{}{0pt}
\pgfsetbuttcap
{
\definecolor{dialinecolor}{rgb}{0.000000, 0.000000, 0.000000}
\pgfsetfillcolor{dialinecolor}
% was here!!!
\definecolor{dialinecolor}{rgb}{0.000000, 0.000000, 0.000000}
\pgfsetstrokecolor{dialinecolor}
\pgfpathmoveto{\pgfpoint{26.800000\du}{5.546870\du}}
\pgfpatharc{360}{181}{0.300000\du and 0.300000\du}
\pgfusepath{stroke}
}
% setfont left to latex
\definecolor{dialinecolor}{rgb}{0.000000, 0.000000, 0.000000}
\pgfsetstrokecolor{dialinecolor}
\node[anchor=west] at (33.500000\du,5.000000\du){$K_i$};
\pgfsetlinewidth{0.100000\du}
\pgfsetdash{}{0pt}
\pgfsetdash{}{0pt}
\pgfsetbuttcap
{
\definecolor{dialinecolor}{rgb}{0.000000, 0.000000, 0.000000}
\pgfsetfillcolor{dialinecolor}
% was here!!!
\definecolor{dialinecolor}{rgb}{0.000000, 0.000000, 0.000000}
\pgfsetstrokecolor{dialinecolor}
\draw (26.500000\du,8.000000\du)--(17.500000\du,9.500000\du);
}
\pgfsetlinewidth{0.100000\du}
\pgfsetdash{}{0pt}
\pgfsetdash{}{0pt}
\pgfsetmiterjoin
\pgfsetbuttcap
{
\definecolor{dialinecolor}{rgb}{0.000000, 0.000000, 0.000000}
\pgfsetfillcolor{dialinecolor}
% was here!!!
\definecolor{dialinecolor}{rgb}{0.000000, 0.000000, 0.000000}
\pgfsetstrokecolor{dialinecolor}
\pgfpathmoveto{\pgfpoint{26.500000\du}{4.500000\du}}
\pgfpathcurveto{\pgfpoint{26.500000\du}{3.900000\du}}{\pgfpoint{22.000000\du}{4.600000\du}}{\pgfpoint{22.000000\du}{4.000000\du}}
\pgfusepath{stroke}
}
\pgfsetlinewidth{0.100000\du}
\pgfsetdash{}{0pt}
\pgfsetdash{}{0pt}
\pgfsetmiterjoin
\pgfsetbuttcap
{
\definecolor{dialinecolor}{rgb}{0.000000, 0.000000, 0.000000}
\pgfsetfillcolor{dialinecolor}
% was here!!!
\definecolor{dialinecolor}{rgb}{0.000000, 0.000000, 0.000000}
\pgfsetstrokecolor{dialinecolor}
\pgfpathmoveto{\pgfpoint{26.500000\du}{4.500000\du}}
\pgfpathcurveto{\pgfpoint{26.500000\du}{4.100000\du}}{\pgfpoint{28.000000\du}{4.600000\du}}{\pgfpoint{28.000000\du}{4.000000\du}}
\pgfusepath{stroke}
}
\pgfsetlinewidth{0.100000\du}
\pgfsetdash{}{0pt}
\pgfsetdash{}{0pt}
\pgfsetmiterjoin
\pgfsetbuttcap
{
\definecolor{dialinecolor}{rgb}{0.000000, 0.000000, 0.000000}
\pgfsetfillcolor{dialinecolor}
% was here!!!
\definecolor{dialinecolor}{rgb}{0.000000, 0.000000, 0.000000}
\pgfsetstrokecolor{dialinecolor}
\pgfpathmoveto{\pgfpoint{17.500000\du}{11.500000\du}}
\pgfpathcurveto{\pgfpoint{17.500000\du}{11.880000\du}}{\pgfpoint{16.000000\du}{11.500000\du}}{\pgfpoint{16.000000\du}{12.000000\du}}
\pgfusepath{stroke}
}
\pgfsetlinewidth{0.100000\du}
\pgfsetdash{}{0pt}
\pgfsetdash{}{0pt}
\pgfsetmiterjoin
\pgfsetbuttcap
{
\definecolor{dialinecolor}{rgb}{0.000000, 0.000000, 0.000000}
\pgfsetfillcolor{dialinecolor}
% was here!!!
\definecolor{dialinecolor}{rgb}{0.000000, 0.000000, 0.000000}
\pgfsetstrokecolor{dialinecolor}
\pgfpathmoveto{\pgfpoint{17.500000\du}{11.500000\du}}
\pgfpathcurveto{\pgfpoint{17.500000\du}{12.100000\du}}{\pgfpoint{22.000000\du}{11.300000\du}}{\pgfpoint{22.000000\du}{12.000000\du}}
\pgfusepath{stroke}
}
\pgfsetlinewidth{0.100000\du}
\pgfsetdash{}{0pt}
\pgfsetdash{}{0pt}
\pgfsetbuttcap
{
\definecolor{dialinecolor}{rgb}{0.000000, 0.000000, 0.000000}
\pgfsetfillcolor{dialinecolor}
% was here!!!
\pgfsetarrowsend{stealth}
\definecolor{dialinecolor}{rgb}{0.000000, 0.000000, 0.000000}
\pgfsetstrokecolor{dialinecolor}
\draw (17.500000\du,9.500000\du)--(17.500000\du,11.500000\du);
}
\definecolor{dialinecolor}{rgb}{1.000000, 1.000000, 1.000000}
\pgfsetfillcolor{dialinecolor}
\fill (21.700000\du,6.175000\du)--(21.700000\du,7.804167\du)--(23.982500\du,7.804167\du)--(23.982500\du,6.175000\du)--cycle;
\pgfsetlinewidth{0.100000\du}
\pgfsetdash{}{0pt}
\pgfsetdash{}{0pt}
\pgfsetmiterjoin
\definecolor{dialinecolor}{rgb}{0.000000, 0.000000, 0.000000}
\pgfsetstrokecolor{dialinecolor}
\draw (21.700000\du,6.175000\du)--(21.700000\du,7.804167\du)--(23.982500\du,7.804167\du)--(23.982500\du,6.175000\du)--cycle;
% setfont left to latex
\definecolor{dialinecolor}{rgb}{0.000000, 0.000000, 0.000000}
\pgfsetstrokecolor{dialinecolor}
\node at (22.841250\du,7.120000\du){S-box};
\pgfsetlinewidth{0.100000\du}
\pgfsetdash{}{0pt}
\pgfsetdash{}{0pt}
\pgfsetmiterjoin
\pgfsetbuttcap
{
\definecolor{dialinecolor}{rgb}{0.000000, 0.000000, 0.000000}
\pgfsetfillcolor{dialinecolor}
% was here!!!
\pgfsetarrowsend{stealth}
{\pgfsetcornersarced{\pgfpoint{0.000000\du}{0.000000\du}}\definecolor{dialinecolor}{rgb}{0.000000, 0.000000, 0.000000}
\pgfsetstrokecolor{dialinecolor}
\draw (26.247900\du,5.500000\du)--(26.247900\du,5.500000\du)--(25.300000\du,5.500000\du)--(25.300000\du,6.455078\du);
}}
\pgfsetlinewidth{0.100000\du}
\pgfsetdash{}{0pt}
\pgfsetdash{}{0pt}
\pgfsetbuttcap
{
\definecolor{dialinecolor}{rgb}{0.000000, 0.000000, 0.000000}
\pgfsetfillcolor{dialinecolor}
% was here!!!
\pgfsetarrowsend{stealth}
\definecolor{dialinecolor}{rgb}{0.000000, 0.000000, 0.000000}
\pgfsetstrokecolor{dialinecolor}
\draw (24.800000\du,7.000000\du)--(23.982500\du,6.989583\du);
}
\pgfsetlinewidth{0.100000\du}
\pgfsetdash{}{0pt}
\pgfsetdash{}{0pt}
\pgfsetmiterjoin
\pgfsetbuttcap
{
\definecolor{dialinecolor}{rgb}{0.000000, 0.000000, 0.000000}
\pgfsetfillcolor{dialinecolor}
% was here!!!
\definecolor{dialinecolor}{rgb}{0.000000, 0.000000, 0.000000}
\pgfsetstrokecolor{dialinecolor}
\pgfpathmoveto{\pgfpoint{17.500000\du}{4.500000\du}}
\pgfpathcurveto{\pgfpoint{17.500000\du}{4.100000\du}}{\pgfpoint{16.000000\du}{4.600000\du}}{\pgfpoint{16.000000\du}{4.000000\du}}
\pgfusepath{stroke}
}
\pgfsetlinewidth{0.100000\du}
\pgfsetdash{}{0pt}
\pgfsetdash{}{0pt}
\pgfsetmiterjoin
\pgfsetbuttcap
{
\definecolor{dialinecolor}{rgb}{0.000000, 0.000000, 0.000000}
\pgfsetfillcolor{dialinecolor}
% was here!!!
\definecolor{dialinecolor}{rgb}{0.000000, 0.000000, 0.000000}
\pgfsetstrokecolor{dialinecolor}
\pgfpathmoveto{\pgfpoint{17.500000\du}{4.500000\du}}
\pgfpathcurveto{\pgfpoint{17.500000\du}{4.000000\du}}{\pgfpoint{22.000000\du}{4.600000\du}}{\pgfpoint{22.000000\du}{4.000000\du}}
\pgfusepath{stroke}
}
\pgfsetlinewidth{0.100000\du}
\pgfsetdash{}{0pt}
\pgfsetdash{}{0pt}
\pgfsetmiterjoin
\definecolor{dialinecolor}{rgb}{1.000000, 1.000000, 1.000000}
\pgfsetfillcolor{dialinecolor}
\fill (18.800000\du,6.487500\du)--(18.800000\du,7.487500\du)--(20.800000\du,7.487500\du)--(20.800000\du,6.487500\du)--cycle;
\definecolor{dialinecolor}{rgb}{0.000000, 0.000000, 0.000000}
\pgfsetstrokecolor{dialinecolor}
\draw (18.800000\du,6.487500\du)--(18.800000\du,7.487500\du)--(20.800000\du,7.487500\du)--(20.800000\du,6.487500\du)--cycle;
% setfont left to latex
\definecolor{dialinecolor}{rgb}{0.000000, 0.000000, 0.000000}
\pgfsetstrokecolor{dialinecolor}
\node[anchor=west] at (18.600000\du,7.000000\du){$<<<$};
\pgfsetlinewidth{0.100000\du}
\pgfsetdash{}{0pt}
\pgfsetdash{}{0pt}
\pgfsetbuttcap
{
\definecolor{dialinecolor}{rgb}{0.000000, 0.000000, 0.000000}
\pgfsetfillcolor{dialinecolor}
% was here!!!
\pgfsetarrowsend{stealth}
\definecolor{dialinecolor}{rgb}{0.000000, 0.000000, 0.000000}
\pgfsetstrokecolor{dialinecolor}
\draw (21.700000\du,6.989583\du)--(20.800000\du,6.987500\du);
}
\pgfsetlinewidth{0.100000\du}
\pgfsetdash{}{0pt}
\pgfsetdash{}{0pt}
\pgfsetbuttcap
\pgfsetmiterjoin
\pgfsetlinewidth{0.100000\du}
\pgfsetbuttcap
\pgfsetmiterjoin
\pgfsetdash{}{0pt}
\definecolor{dialinecolor}{rgb}{1.000000, 1.000000, 1.000000}
\pgfsetfillcolor{dialinecolor}
\pgfpathellipse{\pgfpoint{17.500000\du}{7.000000\du}}{\pgfpoint{0.500000\du}{0\du}}{\pgfpoint{0\du}{0.500000\du}}
\pgfusepath{fill}
\definecolor{dialinecolor}{rgb}{0.000000, 0.000000, 0.000000}
\pgfsetstrokecolor{dialinecolor}
\pgfpathellipse{\pgfpoint{17.500000\du}{7.000000\du}}{\pgfpoint{0.500000\du}{0\du}}{\pgfpoint{0\du}{0.500000\du}}
\pgfusepath{stroke}
\pgfsetbuttcap
\pgfsetmiterjoin
\pgfsetdash{}{0pt}
\definecolor{dialinecolor}{rgb}{0.000000, 0.000000, 0.000000}
\pgfsetstrokecolor{dialinecolor}
\draw (17.500000\du,6.500000\du)--(17.500000\du,7.500000\du);
\pgfsetbuttcap
\pgfsetmiterjoin
\pgfsetdash{}{0pt}
\definecolor{dialinecolor}{rgb}{0.000000, 0.000000, 0.000000}
\pgfsetstrokecolor{dialinecolor}
\draw (17.000000\du,7.000000\du)--(18.000000\du,7.000000\du);
\pgfsetlinewidth{0.100000\du}
\pgfsetdash{}{0pt}
\pgfsetdash{}{0pt}
\pgfsetbuttcap
{
\definecolor{dialinecolor}{rgb}{0.000000, 0.000000, 0.000000}
\pgfsetfillcolor{dialinecolor}
% was here!!!
\pgfsetarrowsend{stealth}
\definecolor{dialinecolor}{rgb}{0.000000, 0.000000, 0.000000}
\pgfsetstrokecolor{dialinecolor}
\draw (18.800000\du,6.987500\du)--(18.000000\du,7.000000\du);
}
\pgfsetlinewidth{0.100000\du}
\pgfsetdash{}{0pt}
\pgfsetdash{}{0pt}
\pgfsetbuttcap
{
\definecolor{dialinecolor}{rgb}{0.000000, 0.000000, 0.000000}
\pgfsetfillcolor{dialinecolor}
% was here!!!
\definecolor{dialinecolor}{rgb}{0.000000, 0.000000, 0.000000}
\pgfsetstrokecolor{dialinecolor}
\draw (17.500000\du,4.500000\du)--(17.500000\du,6.500000\du);
}
\pgfsetlinewidth{0.100000\du}
\pgfsetdash{}{0pt}
\pgfsetdash{}{0pt}
\pgfsetbuttcap
{
\definecolor{dialinecolor}{rgb}{0.000000, 0.000000, 0.000000}
\pgfsetfillcolor{dialinecolor}
% was here!!!
\definecolor{dialinecolor}{rgb}{0.000000, 0.000000, 0.000000}
\pgfsetstrokecolor{dialinecolor}
\draw (17.500000\du,7.500000\du)--(17.500000\du,8.000000\du);
}
\pgfsetlinewidth{0.100000\du}
\pgfsetdash{}{0pt}
\pgfsetdash{}{0pt}
\pgfsetbuttcap
{
\definecolor{dialinecolor}{rgb}{0.000000, 0.000000, 0.000000}
\pgfsetfillcolor{dialinecolor}
% was here!!!
\definecolor{dialinecolor}{rgb}{0.000000, 0.000000, 0.000000}
\pgfsetstrokecolor{dialinecolor}
\draw (17.500000\du,8.000000\du)--(26.500000\du,9.500000\du);
}
\pgfsetlinewidth{0.100000\du}
\pgfsetdash{}{0pt}
\pgfsetdash{}{0pt}
\pgfsetmiterjoin
\pgfsetbuttcap
{
\definecolor{dialinecolor}{rgb}{0.000000, 0.000000, 0.000000}
\pgfsetfillcolor{dialinecolor}
% was here!!!
\definecolor{dialinecolor}{rgb}{0.000000, 0.000000, 0.000000}
\pgfsetstrokecolor{dialinecolor}
\pgfpathmoveto{\pgfpoint{26.500000\du}{11.500000\du}}
\pgfpathcurveto{\pgfpoint{26.500000\du}{12.000000\du}}{\pgfpoint{22.000000\du}{11.300000\du}}{\pgfpoint{22.000000\du}{12.000000\du}}
\pgfusepath{stroke}
}
\pgfsetlinewidth{0.100000\du}
\pgfsetdash{}{0pt}
\pgfsetdash{}{0pt}
\pgfsetmiterjoin
\pgfsetbuttcap
{
\definecolor{dialinecolor}{rgb}{0.000000, 0.000000, 0.000000}
\pgfsetfillcolor{dialinecolor}
% was here!!!
\definecolor{dialinecolor}{rgb}{0.000000, 0.000000, 0.000000}
\pgfsetstrokecolor{dialinecolor}
\pgfpathmoveto{\pgfpoint{26.500000\du}{11.500000\du}}
\pgfpathcurveto{\pgfpoint{26.500000\du}{11.900000\du}}{\pgfpoint{28.000000\du}{11.400000\du}}{\pgfpoint{28.000000\du}{12.000000\du}}
\pgfusepath{stroke}
}
\pgfsetlinewidth{0.100000\du}
\pgfsetdash{}{0pt}
\pgfsetdash{}{0pt}
\pgfsetbuttcap
{
\definecolor{dialinecolor}{rgb}{0.000000, 0.000000, 0.000000}
\pgfsetfillcolor{dialinecolor}
% was here!!!
\pgfsetarrowsend{stealth}
\definecolor{dialinecolor}{rgb}{0.000000, 0.000000, 0.000000}
\pgfsetstrokecolor{dialinecolor}
\draw (26.500000\du,9.500000\du)--(26.500000\du,11.500000\du);
}
\pgfsetlinewidth{0.100000\du}
\pgfsetdash{}{0pt}
\pgfsetdash{}{0pt}
\pgfsetbuttcap
{
\definecolor{dialinecolor}{rgb}{0.000000, 0.000000, 0.000000}
\pgfsetfillcolor{dialinecolor}
% was here!!!
\pgfsetarrowsend{stealth}
\definecolor{dialinecolor}{rgb}{0.000000, 0.000000, 0.000000}
\pgfsetstrokecolor{dialinecolor}
\draw (26.500000\du,7.000000\du)--(25.800000\du,7.000000\du);
}
\end{tikzpicture}

            \caption{Циклова функція ГОСТ~28147-89}
        \end{figure}
    \end{minipage}%
\end{frame}

\begin{frame}[fragile, shrink]{Алгебраїчний криптоаналіз}{Вирішення алгебраїчних рівнянь}
    \small
    \begin{block}{Система рівнянь ГОСТ~28147-89}
        \begin{itemize}
            \item реалізовано з використанням системи символьної алгебри SAGE;
            \item один раунд шифру містить $325$ рівнянь другого степеня;
            \item повний шифр описується $10432$ поліномами від $4416$ змінних;
            \item можливо вирішити алгебраїчну систему для $5$ раундів ГОСТ~28147-89.
        \end{itemize}
    \end{block}

    \begin{block}{Використання CryptoMiniSat}
        \begin{enumerate}
            \item формується система рівнянь шифра в АНФ за допомогою SAGE;
            \item система рівнянь конвертується з АНФ формату до КНФ;
            \item \verb+CryptoMiniSat+ знаходить набір змінних, що задовольняє систему рівнянь. 
        \end{enumerate} 
    \end{block}
    \begin{example}
%        \begin{lstlisting}
%        sage: gost = Gost(rounds=5)
%        sage: f = gost.polynomial_system()
%        sage: solver = ANFSatSolver(f.ring())
%        sage: s, t = solver(f)
%        \end{lstlisting}
    \end{example}
\end{frame}

\begin{frame}{Висновки}

    \begin{block}{Алгебраїчний криптоаналіз}
        \begin{itemize}
            \item алгоритм ГОСТ вдало описується системою рівнянь другого степеня;
            \item на даний момент можливо вирішити алгебраїчну систему, що
                описує 5 раундів шифрування ГОСТ~28147;
            \item подальша оптимізація може дозволити успішно вирішити алгебраїчну систему
                для більшої кількості раундів.
        \end{itemize} 
    \end{block}
\end{frame}

\begin{frame}[allowframebreaks]{Список публікацій}
    \scriptsize
    \begingroup
    \renewcommand{\section}[2]{}%
    \begin{thebibliography}{1}
            \providecommand*{\BibEmph}[1]{#1}
            \providecommand*{\cyrdash}{\hbox to.8em{--\hss--}}
            \providecommand*{\BibDash}{\ifdim\lastskip>0pt\unskip\nobreak\hskip.2em\fi\cyrdash\hskip.2em\ignorespaces}

        \bibitem{Kiyanchuk:DESSERT:2012}
            \BibEmph{Oliynykov~R.~V., Kiyanchuk~R.~I.} 
            \newblock {Perspective Symmetric Block Cipher optimized for Hardware Implementation}~// 
            \newblock {6-th International Conference ``Dependable Systems, Services \& Technologies (DESSERT'12)''}. \BibDash 2012.

        \bibitem{Kiyanchuk:visnyk:2012}
            \BibEmph{Kiyanchuk~R.~I., Oliynykov~R.~V.} \newblock {Linear transformation properties of
            ZUC cipher}~// \newblock \BibEmph{Visnyk}. \BibDash 2012. \BibDash{Mathematical modeling. Information technologies. Computer-aided
            control systems.}

        \bibitem{karazina:zuc}
            \BibEmph{{Kiyanchuk, R. I. and Oliynykov R. V.}} \newblock {Linear transformation
            properties of ZUC cipher}~// \newblock {Computer modeling in high-end technologies}~/
            Kharkiv national university of radio electronics. \BibDash {Kharkiv}, 2012. \BibDash P.~199 -- 202.

        \bibitem{Kiyanchuk:2012:Banking}
            \BibEmph{Кіянчук~Р.~І.} \newblock {Диференційний аналіз S-функцій}~// 
            \newblock {Наукові дослідження молоді вирішенню проблем європейської інтеграції}~/
            Харківський університет банківської справи. \BibDash Харків, 2012.
            \BibDash Електронне видання на CD-ROM.

        \bibitem{Kiyanchuk:2012:MMF}
            \BibEmph{Кіянчук~Р.~І.} \newblock {Диференційний аналіз S-функцій}~// 
            \newblock {Радіоелектроніка та молодь у XXI столітті}~/ Харківський
            національний університет радіоелектроніки. \BibDash Харків, 2012. \BibDash {с.}~130 -- 131.

        \bibitem{Kiyanchuk:2011:MMF}
            \BibEmph{Кіянчук~Р.~І.} \newblock {Порівняльний аналіз
            IDEA-подібних блочних симетричних шифрів}~// \newblock {
            Міжнародна конференція ``Комп'ютерна інженерія''}~/ 
            Харківський національний університет радіоелектроніки. \BibDash
            Харків, 2011. \BibDash {с.}~225 -- 227.

        \bibitem{Kiyanchuk:IREF:2011:present}
            \BibEmph{Олійников~Р.~В., Кіянчук~Р.~І.} 
            \newblock{Перспективний блочний симетричний шифр оптимізований для
            апаратної реалізації}~// 
            \newblock{Міжнародна конференція ``Телекомунікаційні системи та
            технології''}~/ 
            Харківський національний університет радіоелектроніки.
            \BibDash Т.~II. \BibDash {Харків, Україна}, 2011. \BibDash 
            с.~321 -- 330.


        \bibitem{Kiyanchuk:2011:Customs}
            \BibEmph{Олейников, Р.~В. and Киянчук, Р.~И.} 
            \newblock{Использование Т-функций в симметричных криптографических преобразованиях}~// 
            \newblock{Материалы международной научно-практической конференции <<Перспективы
            развития информационных и транспортно-таможенных технологий в таможенном
            деле, внешнеэкономической деятельности и управлении организациями>>}~/ 
            {Харьковский национальный университет радиоэлектроники}. \BibDash Днепропетровск, 2011. \BibDash 
            {c.}~213 -- 215.

        \bibitem{Kiyanchuk:2009:rijndael}
            \BibEmph{Долгов, В.~И. and Лисицкая, И.~В. and Киянчук, Р.~И.} 
            \newblock {RIJNDAEL -- это новое или хорошо забытое старое?}~// 
            \newblock{Компьютерные Науки и Технологии}~/ \BibDash
            2009. \BibDash с.~32 -- 35.
    \end{thebibliography}
\endgroup
\end{frame} 
\end{document}

