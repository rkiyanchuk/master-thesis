% AUTHOR: Ruslan Kiianchuk <ruslan.kiianchuk@gmail.com>

\Chapter{Conclusions}
\label{sec:conclusions}

As global shift to portable devices usage spawned demand for very efficient and
also secure cryptographic primitives, efficient methods for comprehensive
security evaluation of perspective ciphers are required. Current usage of
insecure ciphers in various fields of information technologies proves the
importance of carefull ciphers security evaluation before deploying them into
real systems.

The accomplished work resulted in development of methods for defining most
widely used cryptographic primitives with system of non-linear equations. The
techniques of obtaining equations for bit permutations, modular addition, some
logical operations and S-boxes should allow to construct full-scale non-linear
equations systems for most modern symmetric ciphers. Best approaches for solving
the obtained equations sets are described and allow to solve reduced round
versions of analyzed ciphers, research algebraic properties of individual
transformations on low-end computers. Usage of resources with more
computational power will increase feasibility of full-scale cipher analysis.

As the result of the work software tools for computational algebra that provide
needed functionality are described and reference implementation for defining
individual transformations with non-linear equations and constructing full-scale
system of equations for modern symmetric ciphers is provided.

An algebraic equation system describing \gost\ cipher and obtained using the
suggested method contains $10432$ polynomials in $4416$ variables. \misty\
cipher is described with $8448$ equations in $3680$ variables using the same
method. Number of equations and variables in \gost system is resembling
that of PRESENT and \misty equations set is larger than that of AES ($3680$
variables in \misty\ against $1600$ variables in AES).

Using the described techniques it is possible to solve a $6$ round
\gost\ polynomial system with $4$ pairs of plaintexts and ciphertexts at the
moment. Thereby the reduced \gost\ algorithm using $160$ out of $256$ key bits is
broken by an algebraic attack. Such statistics strengthens the opinion about
AES vulnerability to algebraic attacks. The
algebraic attack on \misty\ is performed. The equations system for two cipher
round could be solved and equivalent keys for a given
\mbox{plaintext/ciphertext} pair could be found for up to $4$ rounds.
Solving these systems of equations for additional rounds requires more
computation power, however finding the solution may be possible on more
efficient hi-end computers. All computations have been executed on
 \verb+Intel Core i5-3570+ CPU at 3.40~GHz with 8~Gb RAM.

The nondetermination of SAT-solver algorithms do not allow to bind
characteristics of obtained equations set (like its degree, number of equations
and variables, etc.) to some certain time complexity of solving the given
system. However these factors may be used to estimate the feasibility of
solving the system and rationality of allocating processing time for analysis.

Also algebraic analysis in combination with other known cryptanalytic methods
(linear, differential, integral, etc.) proved to be efficient enough for
security evaluation of a cipher~\cite{Albrecht2010}. Considering this practice
algebraic analysis may increase the significance of investigating baby-ciphers
that are shrinked versions of original cryptoalgorithms, so such approach is a
subject for future researches.

\osh{
For labour protection the employee working conditions are analysed for their
compliance with normative documents on safety engineering and sanitaion.
Harmful and dangerous production factors are retrieved and evaluated using the
built ``Human--Machine--Environment'' interaction system. Corresponding safety
measures are developed in order to provide favourable working conditions.
}
