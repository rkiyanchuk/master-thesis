% AUTHOR: Ruslan Kiianchuk <ruslan.kiianchuk@gmail.com>

\Chapter{Introduction}
\label{sec:intro}

Symmetric cryptographic transformations are known to be the only effective 
method of providing data confidentiality and integrity in all fields of
communication technologies~\cite{moldovyan2007innovative}. They need to be not
only cryptographically strong, but also have high performance and low resources
consumption to satisfy modern needs for securing information. Consequently,
robust requirements on cryptographic security, lightweight implementation and
performance are entrusted to such transformations.

Symmetric ciphers came through a long history of development and improvement
from the most primitive schemes based on symbols substitution and disk
encryptors to advanced mathematical algorithms following Kerckhoffs'
principle~\cite{kahn1996codebreakers}.
A tremendous contribution to enciphering theory has been done by Claude
Shannon's work~\cite{shannon:secrecy} back in 1949. He introduced the
fundamentals of information theory and made it possible to evaluate and
mathematically prove cipher security. Ubiquitous computerization, mass
deployment of pervasive devices and extensive Internet access caused
cryptography to find comprehensive applications in information systems. Despite
the advanced mathematics behind modern cryptoalgorithms, real operating
security systems often have weaknesses due to incorrect usage or implementation
errors.

Security of mobile communication systems fell far behind from what was
state-of-the-art in modern cryptography. The A5/1 cipher used in GSM standard
for over 10 years can be broken within seconds using a combined distributed
rainbow table code book to decrypt GSM voice calls and text
messages~\cite{secproject}.
Communication over satellite phones has also been shown to be insecure after
reverse engineering the proprietary ciphers \mbox{GMR-1} and \mbox{GMR-2}.
\mbox{GMR-1} is a variant of A5/2 cipher (which is prohibited for
implementation in mobile phones as of July 2007) and is vulnerable to a known
ciphertext-only attack with an average case complexity of
$2^{32}$~\cite{3gpp:a52:2007}.
\mbox{GMR-2} is an original cipher, but its session key can be recovered with
65 bytes of keystream at a moderate computational
complexity~\cite{kiyanchuk:zuc}. 

In~\cite{cryptoeprint-2010-013} an effective attack on KASUMI cipher
\footnote{KASUMI cipher is adopted by 3GPP as A5/3 cryptographic algorithm
for securing mobile communications.} 
used in 3G systems is presented. Key recovery for the full cipher was done in
hours on low-end computer that used unoptimized cipher reference implementation.
It is worth noting that KASUMI is based on MISTY cryptoalgorithm which however 
could not be broken by the same attack. Therefore even slight modifications to
robust cipher may significantly impact the security of the algorithm.

The need of deep analysis and public evaluation of cryptographic algorithms
before deploying them into real systems is obvious. Several ciphers are
considered for becoming worldwide standards in the field of providing data
confidentiality at the moment. 

\misty\ is a symmetric block cipher designed in 1995 which has been one of the
selected algorithms in the European NESSIE project~\cite{Preneel_neweuropean}
and recommended for Japaneese government use~\cite{cryptrec:misty}. Though no
vulnerabilities have yet been found in \misty\ cipher, some of its successors
were broken (KASUMI) or provided with a theoretical attack (Camellia) which may
be feasible in future with increase of computations
power~\cite{Biryukov03decanniere}.

\gost\ is a legacy cipher that has been a subject to cryptanalysis for more than
20 years. Despite its wide usage in Ukraine and other CIS countries since the
publication in 1990, the cipher has been proposed for international
standardization only in 2010, but hasn't been accepted
however~\cite{isoiec-18033}. Therefore a detailed security evaluation and
properties analysis of these ciphers are essential to validate their pervasive
deployment into security systems.

\osh{
    Chapter~\ref{sec:labour} analyses PC users working conditions and
    their compliance with normative documents on safety engineering  and
    sanitation. For retrieving and evaluating the influence of possible dangerous
    or harmful production factors an interaction system
    ``Human--Machine--Environment'' (HME) is developed. Safety measures are
    developed as the result of such system analysis.
}
