% AUTHOR: Ruslan Kiianchuk <ruslan.kiianchuk@gmail.com>

\selectlanguage{ukrainian}
\begin{abstract}

    Магістерська робота містить \pageref{LastPage}~сторінок, \totfig~рисунків,
    \tottab~таблиці, \totapp~додатків та \totref~джерело.

    У роботі представлено аналіз перспективних симетричних шифрів, що є
    стандартами на державному та міжнародному рівні.

    Розроблено методи побудови системи нелінійних рівнянь низького степеня від
    багатьох невідомих, що описують криптоалгоритми \misty\ та
    ГОСТ~28147-89. Представлено характеристики алгебраїчної системи 
    рівнянь кожного шифру та їх порівняння з аналогічними системами рівнянь 
    для криптоалгоритмів AES, Camellia та PRESENT.

    Оцінено криптографічну стійкість шифрів \mbox{ГОСТ~28147-89} та \misty\ до
    алгебраїчного криптоаналізу. Здійснено алгебраїчну атаку на зменшені версії
    криптоалгоритмів використовуючи методи SAT-solver для вирішення системи
    нелінійних рівнянь та відновлення ключа шифрування.

    \keywords{симетричні шифри, \misty, ГОСТ~28147-89, алгебраїчний криптоаналіз}
\end{abstract}

\clearpage
\selectlanguage{russian}
\begin{abstract}

    Магистерская работа содержит \pageref{LastPage}~страниц, \totfig~рисунков,
    \tottab~таблицы, \totapp~приложений и \totref~источник.

    В работе представлено анализ перспективных симетричных шифров, которые
    являются стандартами на государственном и международном уровне.

    Разработано методы построение системы нелинейных уравнений низкой степени от
    многих переменных, которые описывают криптографические алгоритмы \misty\ и 
    ГОСТ~28147-89. Представлено характеристики алгебраической системы уравений
    каждого шифра и их сравнение с аналогичными системами уравнений для
    криптоалгоритмов AES, Camellia, PRESENT.

    Оценено криптографическую стойкость алгоритмов \mbox{ГОСТ~28147-89} и \misty\ к
    алгебраическому криптоанализу. Выполнено алгебраическую атаку на уменьшенные
    версии криптоалгоритмов, используя методы SAT-solver для решения системы
    нелинейных уравнений и восстановления ключа шифрования.

    \keywords{Симметричные шифры, \misty, ГОСТ~28147-89, \\ алгебраический криптоанализ}
\end{abstract}

\clearpage
\selectlanguage{english}
\begin{abstract}

    This thesis contains \pageref{LastPage}~pages, \totfig~figures, 
    \tottab~tables, \totapp~appendices and \totref~references.

    The work presents analysis of symmetric block ciphers that are adopted
    standards on country and international levels.

    Methods for constructing non-linear multivariate quadratic (MQ) equations
    systems that define cryptoalgorithms \misty\ and \gost\ are developed.
    Characteristics for each algebraic system are presented and compared to
    analogous systems for cryptoalgorithms AES, DES and PRESENT.

    Further the strength of \gost\ and \misty\ ciphers to algebraic
    cryptanalysis is researched. Algebraic attack on reduced rounds versions of
    the ciphers is executed using SAT-solver methods for solving non-linear
    equations systems and recovering the enciphering key.

    \keywords{symmetric ciphers, algebraic cryptanalysis, \misty, \gost}
\end{abstract}
\clearpage
